% Syllabus Template from Arman Shokrollahi
% https://www.overleaf.com/latex/templates/syllabus-template-course-info/gbqbpcdgvxjs

\documentclass[11pt, letterpaper]{article}
%\usepackage{geometry}
\usepackage[inner=2cm,outer=2cm,top=2.5cm,bottom=2.5cm]{geometry}
\pagestyle{empty}
\usepackage{graphicx}
\usepackage{fancyhdr, lastpage, bbding, pmboxdraw}
\usepackage[usenames,dvipsnames]{color}
\definecolor{darkblue}{rgb}{0,0,.6}
\definecolor{darkred}{rgb}{.7,0,0}
\definecolor{darkgreen}{rgb}{0,.6,0}
\definecolor{red}{rgb}{.98,0,0}
\usepackage[colorlinks,pagebackref,pdfusetitle,urlcolor=darkblue,citecolor=darkblue,linkcolor=darkred,bookmarksnumbered,plainpages=false]{hyperref}
\renewcommand{\thefootnote}{\fnsymbol{footnote}}

\pagestyle{fancyplain}
\fancyhf{}
\lhead{ \fancyplain{}{Causal Inference} }
%\chead{ \fancyplain{}{} }
\rhead{ \fancyplain{}{Spring 2023} }%\today
%\rfoot{\fancyplain{}{page \thepage\ of \pageref{LastPage}}}
\fancyfoot[RO, LE] {page \thepage\ of \pageref{LastPage} }
\thispagestyle{plain}

%%%%%%%%%%%% LISTING %%%
\usepackage{listings}
\usepackage{caption}
\usepackage{setspace}
\DeclareCaptionFont{white}{\color{white}}
\DeclareCaptionFormat{listing}{\colorbox{gray}{\parbox{\textwidth}{#1#2#3}}}
\captionsetup[lstlisting]{format=listing,labelfont=white,textfont=white}
\usepackage{verbatim} % used to display code
\usepackage{fancyvrb}
\usepackage{acronym}
\usepackage{amsthm}
\VerbatimFootnotes % Required, otherwise verbatim does not work in footnotes!



\definecolor{OliveGreen}{cmyk}{0.64,0,0.95,0.40}
\definecolor{CadetBlue}{cmyk}{0.62,0.57,0.23,0}
\definecolor{lightlightgray}{gray}{0.93}



\lstset{
%language=bash,                          % Code langugage
basicstyle=\ttfamily,                   % Code font, Examples: \footnotesize, \ttfamily
keywordstyle=\color{OliveGreen},        % Keywords font ('*' = uppercase)
commentstyle=\color{gray},              % Comments font
numbers=left,                           % Line nums position
numberstyle=\tiny,                      % Line-numbers fonts
stepnumber=1,                           % Step between two line-numbers
numbersep=5pt,                          % How far are line-numbers from code
backgroundcolor=\color{lightlightgray}, % Choose background color
frame=none,                             % A frame around the code
tabsize=2,                              % Default tab size
captionpos=t,                           % Caption-position = bottom
breaklines=true,                        % Automatic line breaking?
breakatwhitespace=false,                % Automatic breaks only at whitespace?
showspaces=false,                       % Dont make spaces visible
showtabs=false,                         % Dont make tabls visible
columns=flexible,                       % Column format
morekeywords={__global__, __device__},  % CUDA specific keywords
}

%%%%%%%%%%%%%%%%%%%%%%%%%%%%%%%%%%%%
\begin{document}
\begin{center}
{\Large \textsc{POLS 8500: Causal Inference}}
\end{center}
\begin{center}
{\large Spring 2023}
\end{center}

\begin{center}
\rule{6.5in}{0.4pt}
\begin{minipage}[t]{.96\textwidth}
\begin{tabular}{llcccll}
\textbf{Professor:} & Joe Ornstein & & &  & \textbf{Time:} & Th 3:55pm -- 6:40pm \\
\textbf{Email:} &  \href{mailto:jornstein@uga.edu}{jornstein@uga.edu} & & & & \textbf{Place:} & 101B Baldwin Hall\\
\textbf{Website:} & \href{https://github.com/joeornstein/causal-inference}{github.com/joeornstein/causal-inference} & & & & &
\end{tabular}
\end{minipage}
\rule{6.5in}{0.4pt}
\end{center}
\vspace{.15cm}
\setlength{\unitlength}{1in}
\renewcommand{\arraystretch}{2}

Correlations alone are rarely satisfying. As social scientists, we don't just want to identify patterns in data -- we want to convincingly explain \textit{why} those patterns exist. In this class, we introduce the workhorse methods for inferring causal relationships from observational data, with a focus on building the skills you need to develop convincing research designs, identify useful variation in your data, and write error-free, reproducible code.

\begin{figure}[h]
	\centering
	\href{https://xkcd.com/552/}{\includegraphics[width = 0.7\textwidth]{img/correlation.png}}
\end{figure}

\section*{Requirements}

Your course grade will be based on three components.

\begin{itemize}
	\item \textbf{Readings and Annotations (25\%):} Each week, read the assigned chapters/articles and annotate them using \href{https://hypothes.is/groups/Qm2VnAae/causal-inference}{Hypothesis}. This is a seminar-style course, so class time will be devoted to discussion and practice, not lectures. It is your responsibility to learn the material through reading, and use Hypothesis to flag passages that are confusing and would benefit from further discussion in class.
	\item \textbf{Replications (2 $\times$ 25\%):} Choose two of \textit{Applications} papers on the syllabus below (each student must choose a different set of papers). Using their posted replication materials, recreate the paper's key results in a Quarto report. You will present your findings in a Mock Conference Presentation during the second week of the module.
	\item \textbf{Final Paper (25\%):} Write a final paper using one or more of the designs we introduced in class. This could be either an original research project or a re-analysis and extension of one of the papers you replicated.
\end{itemize}


\section*{Course Outline}

Here is the rough plan of action for our semester together. Readings abbreviated ``NHK'' denote our primary textbook, Nick Huntington-Klein's \textit{The Effect}, available in hard copy or \href{https://theeffectbook.net/}{free online}. (You should also check out the author's \href{https://www.youtube.com/playlist?list=PLcTBLulJV_AK1hKtnO0-kYrU0D09K-kj8}{YouTube channel} and collection of \href{https://github.com/NickCH-K/TheEffectAssignments}{homework assignments} that accompany the book.)

\subsection*{Week 1: Garbage Cans, Causal Salads, and the Seven Deadly Sins}

\textit{Required Reading:}

\begin{itemize}
	\item Angrist, Joshua D, and Jörn-Steffen Pischke. 2010. ``\href{https://pubs.aeaweb.org/doi/pdfplus/10.1257/jep.24.2.3}{The Credibility Revolution in Empirical Economics: How Better Research Design Is Taking the Con out of Econometrics.}'' \textit{Journal of Economic Perspectives} 24(2): 3–30.
	\item McElreath, Richard. 2021. \href{https://elevanth.org/blog/2021/06/15/regression-fire-and-dangerous-things-1-3/}{\textit{Regression, Fire, and Dangerous Things.}}
	\item Schrodt, Philip A. 2014. ``\href{https://journals.sagepub.com/doi/pdf/10.1177/0022343313499597}{Seven Deadly Sins of Contemporary Quantitative Political Analysis.}'' \textit{Journal of Peace Research} 51(2): 287–300.
\end{itemize}

\subsection*{Week 2: The Fundamental Problem}

\textit{Required Reading:}

\begin{itemize}
	\item NHK Chapters 1-4
\end{itemize}

\noindent \textit{Deeper Dives:}

\begin{itemize}
	\item Abadie, Alberto, Susan Athey, Guido W. Imbens, and Jeffrey M. Wooldridge. 2020. ``Sampling‐Based versus Design‐Based Uncertainty in Regression Analysis.'' \textit{Econometrica} 88(1): 265–96.
\end{itemize}

\subsection*{Weeks 3: D'ya Like DAGs?}

\textit{Required Reading:}

\begin{itemize}
	\item NHK Chapters 5-8
\end{itemize}

\noindent \textit{Deeper Dives:}

\begin{itemize}
	\item Pearl, Judea, and Dana Mackenzie. \textit{The Book of Why: The New Science of Cause and Effect. Penguin Science}. London: Penguin Books, 2019.
	\item Acharya, Avidit, Matthew Blackwell, and Maya Sen. 2016. ``Explaining Causal Findings Without Bias: Detecting and Assessing Direct Effects.'' American Political Science Review 110(3): 512–29.
	\item Montgomery, Jacob M., Brendan Nyhan, and Michelle Torres. ``How Conditioning on Posttreatment Variables Can Ruin Your Experiment and What to Do about It.'' \textit{American Journal of Political Science} 62, no. 3 (2018): 760–75.
\end{itemize}

\subsection*{Weeks 4--5: Experiments (Natural and Unnatural)}

\textit{Required Reading:}

\begin{itemize}
	\item NHK Chapters 9-11

\end{itemize}

\noindent \textit{Applications:}

\begin{itemize}
	\item Hall, Andrew B., Connor Huff, and Shiro Kuriwaki. 2019. ``Wealth, Slaveownership, and Fighting for the Confederacy: An Empirical Study of the American Civil War.'' \textit{American Political Science Review} 113(3): 658–73.
	\item Green, Donald P., Tiffany C. Davenport, and Kolby Hanson. 2019. ``Are There Long-Term Effects of the Vietnam Draft on Political Attitudes or Behavior? Apparently Not.'' \textit{Journal of Experimental Political Science} 6(02): 71–80.
	\item Loewen, Peter John, Royce Koop, Jaime Settle, and James H. Fowler. 2014. “A Natural Experiment in Proposal Power and Electoral Success.” \textit{American Journal of Political Science} 58(1): 189–96.
\end{itemize}

\noindent \textit{Deeper Dives:}

\begin{itemize}
	\item Glynn, Adam N., and Konstantin Kashin. 2018. ``Front-Door Versus Back-Door Adjustment With Unmeasured Confounding: Bias Formulas for Front-Door and Hybrid Adjustments With Application to a Job Training Program.'' \textit{Journal of the American Statistical Association} 113(523): 1040–49.
	\item Sekhon, Jasjeet S., and Rocío Rocio Titiunik. 2012. ``When Natural Experiments Are Neither Natural nor Experiments.'' \textit{American Political Science Review} 106(1): 35–57.
\end{itemize}

\subsection*{Weeks 6--7: Selection on Observables (Regression and Matching)}

\textit{Required Reading:}

\begin{itemize}
	\item NHK Chapter 12-14
\end{itemize}

\noindent \textit{Applications:}

\begin{itemize}
	\item Boyd, Christina L., Lee Epstein, and Andrew D. Martin. 2010. ``Untangling the Causal Effects of Sex on Judging.'' \textit{American Journal of Political Science} 54(2): 389–411.
	\item Mendelberg, Tali, Katherine T. McCabe, and Adam Thal. 2017. ``College Socialization and the Economic Views of Affluent Americans.'' \textit{American Journal of Political Science} 61(3): 606–23.
	\item Black, Ryan C., and Ryan J. Owens. 2016. “Courting the President: How Circuit Court Judges Alter Their Behavior for Promotion to the Supreme Court.” \textit{American Journal of Political Science} 60(1): 30–43.
\end{itemize}


\noindent \textit{Deeper Dives:}

\begin{itemize}
	\item Cinelli, Carlos, and Chad Hazlett. ``Making Sense of Sensitivity: Extending Omitted Variable Bias.'' \textit{Journal of the Royal Statistical Society: Series B (Statistical Methodology)} 82, no. 1 (2020): 39–67. https://doi.org/10.1111/rssb.12348.
	\item Hainmueller, Jens. 2012. ``Entropy Balancing for Causal Effects: A Multivariate Reweighting Method to Produce Balanced Samples in Observational Studies.'' \textit{Political Analysis} 20(1): 25–46.
	\item Xu, Yiqing, and Eddie Yang. 2022. ``Hierarchically Regularized Entropy Balancing.'' \textit{Political Analysis}: 1–8.
	\item King, Gary, Christopher Lucas, and Richard A. Nielsen. 2017. ``The Balance-Sample Size Frontier in Matching Methods for Causal Inference.'' \textit{American Journal of Political Science} 61(2): 473–89.
	\item King, Gary, and Richard Nielsen. 2019. ``Why Propensity Scores Should Not Be Used for Matching.'' \textit{Political Analysis}: 1–20.
	\item Ho, Daniel E., Kosuke Imai, Gary King, and Elizabeth A. Stuart. 2007. “Matching as Nonparametric Preprocessing for Reducing Model Dependence in Parametric Causal Inference.” \textit{Political Analysis} 15(3): 199–236.
\end{itemize}

\subsection*{Weeks 8--9: Fixed Effects}

\textit{Required Readings:}

\begin{itemize}
	\item NHK Chapter 16
\end{itemize}

\noindent \textit{Applications:}

\begin{itemize}
	\item Stokes, Leah C. 2016. ``Electoral Backlash against Climate Policy: A Natural Experiment on Retrospective Voting and Local Resistance to Public Policy.'' \textit{American Journal of Political Science} 60(4): 958–74.
	\item Scott, Ralph. 2022. “Does University Make You More Liberal? Estimating the within-Individual Effects of Higher Education on Political Values.” \textit{Electoral Studies} 77.
	\item Moskowitz, Daniel J. 2021. “Local News, Information, and the Nationalization of U.S. Elections.” \textit{American Political Science Review} 115(1): 114–29.
	\item Hainmueller, Jens, and Dominik Hangartner. 2019. “Does Direct Democracy Hurt Immigrant Minorities? Evidence from Naturalization Decisions in Switzerland.” \textit{American Journal of Political Science} 63(3): 530–47.
\end{itemize}

\noindent \textit{Deeper Dives:}

\begin{itemize}
	\item Imai, Kosuke, and In Song Kim. 2019. ``When Should We Use Unit Fixed Effects Regression Models for Causal Inference with Longitudinal Data?'' \textit{American Journal of Political Science} 63(2): 467–90.
	\item Imai, Kosuke, and In Song Kim. ``On the Use of Two-Way Fixed Effects Regression Models for Causal Inference with Panel Data.'' \textit{Political Analysis} 29, no. 3 (July 2021): 405–15.
	\item Plümper, Thomas, and Vera E Troeger. 2018. “Not so Harmless After All: The Fixed-Effects Model.” \textit{Political Analysis} 27(1): 21–45.
	\item Hazlett, Chad, and Leonard Wainstein. ``Understanding, Choosing, and Unifying Multilevel and Fixed Effect Approaches.'' \textit{Political Analysis} 30, no. 1 (January 2022): 46–65.
	\item Bell, Andrew, and Kelvyn Jones. 2015. “Explaining Fixed Effects: Random Effects Modeling of Time-Series Cross-Sectional and Panel Data.” \textit{Political Science Research and Methods} 3(1): 133–53.
\end{itemize}

\subsection*{Weeks 10--11: Difference-in-Differences}

\textit{Required Reading:}

\begin{itemize}
	\item NHK Chapters 17.1 and 18
\end{itemize}

\noindent \textit{Applications:}

\begin{itemize}
	\item Hall, Andrew B., and Jesse Yoder. 2022. ``Does Homeownership Influence Political Behavior? Evidence from Administrative Data.'' \textit{The Journal of Politics} 84(1): 351–66.
	\item Knutsen, Carl Henrik, Andreas Kotsadam, Eivind Hammersmark Olsen, and Tore Wig. 2017. ``Mining and Local Corruption in Africa.'' \textit{American Journal of Political Science} 61(2): 320–34.
	\item Kuipers, Nicholas, and Alexander Sahn. 2022. ``The Representational Consequences of Municipal Civil Service Reform.'' \textit{American Political Science Review}: 1–17.
	\item Paglayan, Agustina S. 2019. ``Public-Sector Unions and the Size of Government.'' \textit{American Journal of Political Science} 63(1): 21–36.
	\item Paglayan, Agustina S. 2021. ``The Non-Democratic Roots of Mass Education: Evidence from 200 Years.'' \textit{American Political Science Review} 115(1): 179–98.
	\item Martin, Gregory J., and Joshua McCrain. 2019. “Local News and National Politics.” \textit{American Political Science Review} 113(2): 372–84.
\end{itemize}

\noindent \textit{Deeper Dives:}

\begin{itemize}
	\item Roth, Jonathan, Pedro H. C. Sant’Anna, Alyssa Bilinski, and John Poe. ``What’s Trending in Difference-in-Differences? A Synthesis of the Recent Econometrics Literature.'' \textit{arXiv}, January 13, 2022. http://arxiv.org/abs/2201.01194.
	\item Egami, Naoki, and Soichiro Yamauchi. 2022. ``Using Multiple Pretreatment Periods to Improve Difference-in-Differences and Staggered Adoption Designs.'' \textit{Political Analysis}: 1–18.
	\item \href{https://www.youtube.com/user/ariesxyq/videos}{Causal Inference w/ Panel Data Lecture Series (Yiqing Xu)}
	\item \href{https://bcallaway11.github.io/did/articles/did-basics.html}{The \texttt{did} package (Brant Callaway)}
\end{itemize}

\subsection*{Weeks 12--13: Instrumental Variables}

\textit{Required Reading:}

\begin{itemize}
	\item NHK Chapter 19
\end{itemize}

\noindent\textit{Applications:}

\begin{itemize}
	\item Miguel, Edward, Shanker Satyanath, and Ernest Sergenti. 2004. ``Economic Shocks and Civil Conflict: An Instrumental Variables Approach.'' \textit{Journal of Political Economy} 112(4): 725–53.
	\item Trounstine, Jessica. 2015. ``Segregation and Inequality in Public Goods.'' \textit{American Journal of Political Science} 60(3): 709–25.
	\item Scheve, Kenneth, and Theo Serlin. 2022. ``The German Trade Shock and the Rise of the Neo-Welfare State in Early Twentieth-Century Britain.'' \textit{American Political Science Review}: 1–18.
	\item Baccini, Leonardo, and Stephen Weymouth. 2021. ``Gone For Good: Deindustrialization, White Voter Backlash, and US Presidential Voting.'' \textit{American Political Science Review} 115(2): 550–67.
	\item Fouka, Vasiliki, Soumyajit Mazumder, and Marco Tabellini. 2022. ``From Immigrants to Americans: Race and Assimilation during the Great Migration.'' \textit{The Review of Economic Studies} 89(2): 811–42.
\end{itemize}

\noindent \textit{Deeper Dives:}

\begin{itemize}
	\item Mellon, Jonathan. 2020. ``Rain, Rain, Go Away: 137 Potential Exclusion-Restriction Violations for Studies Using Weather as an Instrumental Variable.'' \textit{Working Paper}.
	\item Lal, Apoorva, Yiqing Xu, Mac Lockhart, and Ziwen Zu. ``How Much Should We Trust Instrumental Variable Estimates in Political Science? Practical Advice Based on 57 Replicated Studies.'' \textit{Working Paper}.
\end{itemize}


\subsection*{Week 14-15: Regression Discontinuity}

\textit{Required Reading:}

\begin{itemize}
	\item NHK Chapter 20
\end{itemize}


\noindent \textit{Applications:}

\begin{itemize}
	\item Klašnja, Marko, and Rocío Titiunik. 2017. ``The Incumbency Curse: Weak Parties, Term Limits, and Unfulfilled Accountability.'' \textit{The American Political Science Review} 111(1): 129–48.
	\item Keele, Luke J., and Rocío Titiunik. 2015. ``Geographic Boundaries as Regression Discontinuities.'' \textit{Political Analysis} 23(1): 127–55.
	\item Mummolo, Jonathan. 2018. ``Modern Police Tactics, Police-Citizen Interactions, and the Prospects for Reform.'' \textit{The Journal of Politics} 80(1): 1–15.
	\item Hall, Andrew. 2015. ``What Happens When Extremists Win Primaries?'' \textit{American Political Science Review} 109(1): 1–46.
	\item Apfeld, Brendan, Emanuel Coman, John Gerring, and Stephen Jessee. 2022. ``The Impact of University Attendance on Partisanship.'' \textit{Political Science Research and Methods}: 1–14.
\end{itemize}

\noindent \textit{Deeper Dives:}

\begin{itemize}
	
	\item Cattaneo, Matias D, Nicolas Idrobo, and Rocío Titiunik. 2018. ``A Practical Introduction to Regression Discontinuity Designs: Volume I.''
	\item Marshall, John. 2022. ``Can Close Election Regression Discontinuity Designs Identify Effects of Winning Politician Characteristics?'' \textit{American Journal of Political Science}.
	\item Hartman, Erin. 2021. ``Equivalence Testing for Regression Discontinuity Designs.'' \textit{Political Analysis} 29(4): 505–21.
	\item Ornstein, Joseph T. \& Duck-Mayr, JBrandon. 2022. ``Gaussian Process Regression Discontinuity''. \textit{Working Paper}.
\end{itemize}

\subsection*{Week 16: Bonus Week}

\textit{Readings TBD}


\section*{Academic Honesty}

Remember that when you joined the University of Georgia community, you agreed to abide by a code of conduct outlined in the academic honesty policy called \href{https://honesty.uga.edu/Academic-Honesty-Policy/Introduction/}{\textit{A Culture of Honesty}}. It has some pretty specific things to say on the subject of cheating. Quite specific. I will report any and all dishonest conduct to the Office of the Vice President for Instruction.

\section*{Mental Health and Wellness Resources}

\begin{itemize}
	\item If you or someone you know needs assistance, you are encouraged to contact Student Care and Outreach in the Division of Student Affairs at 706-542-7774 or visit \href{https://sco.uga.edu}{https://sco.uga.edu}. They will help you navigate any difficult circumstances you may be facing by connecting you with the appropriate resources or services. 
	\item UGA has several resources for a student seeking \href{https://www.uhs.uga.edu/bewelluga/bewelluga}{mental health services} or \href{https://www.uhs.uga.edu/info/emergencies}{crisis support}. 
	\item If you need help managing stress anxiety, relationships, etc., please visit \href{https://www.uhs.uga.edu/bewelluga/bewelluga}{BeWellUGA} for a list of FREE workshops, classes, mentoring, and health coaching led by licensed clinicians and health educators in the University Health Center.
	\item Additional resources can be accessed through the UGA App.
\end{itemize}


%%%%%% THE END 
\end{document} 