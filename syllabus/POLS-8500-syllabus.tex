% Syllabus Template from Arman Shokrollahi
% https://www.overleaf.com/latex/templates/syllabus-template-course-info/gbqbpcdgvxjs

\documentclass[11pt, letterpaper]{article}
%\usepackage{geometry}
\usepackage[inner=2cm,outer=2cm,top=2.5cm,bottom=2.5cm]{geometry}
\pagestyle{empty}
\usepackage{graphicx}
\usepackage{fancyhdr, lastpage, bbding, pmboxdraw}
\usepackage[usenames,dvipsnames]{color}
\definecolor{darkblue}{rgb}{0,0,.6}
\definecolor{darkred}{rgb}{.7,0,0}
\definecolor{darkgreen}{rgb}{0,.6,0}
\definecolor{red}{rgb}{.98,0,0}
\usepackage[colorlinks,pagebackref,pdfusetitle,urlcolor=darkblue,citecolor=darkblue,linkcolor=darkred,bookmarksnumbered,plainpages=false]{hyperref}
\renewcommand{\thefootnote}{\fnsymbol{footnote}}

\pagestyle{fancyplain}
\fancyhf{}
\lhead{ \fancyplain{}{Causal Inference} }
%\chead{ \fancyplain{}{} }
\rhead{ \fancyplain{}{Spring 2023} }%\today
%\rfoot{\fancyplain{}{page \thepage\ of \pageref{LastPage}}}
\fancyfoot[RO, LE] {page \thepage\ of \pageref{LastPage} }
\thispagestyle{plain}

%%%%%%%%%%%% LISTING %%%
\usepackage{listings}
\usepackage{caption}
\usepackage{setspace}
\DeclareCaptionFont{white}{\color{white}}
\DeclareCaptionFormat{listing}{\colorbox{gray}{\parbox{\textwidth}{#1#2#3}}}
\captionsetup[lstlisting]{format=listing,labelfont=white,textfont=white}
\usepackage{verbatim} % used to display code
\usepackage{fancyvrb}
\usepackage{acronym}
\usepackage{amsthm}
\VerbatimFootnotes % Required, otherwise verbatim does not work in footnotes!



\definecolor{OliveGreen}{cmyk}{0.64,0,0.95,0.40}
\definecolor{CadetBlue}{cmyk}{0.62,0.57,0.23,0}
\definecolor{lightlightgray}{gray}{0.93}



\lstset{
%language=bash,                          % Code langugage
basicstyle=\ttfamily,                   % Code font, Examples: \footnotesize, \ttfamily
keywordstyle=\color{OliveGreen},        % Keywords font ('*' = uppercase)
commentstyle=\color{gray},              % Comments font
numbers=left,                           % Line nums position
numberstyle=\tiny,                      % Line-numbers fonts
stepnumber=1,                           % Step between two line-numbers
numbersep=5pt,                          % How far are line-numbers from code
backgroundcolor=\color{lightlightgray}, % Choose background color
frame=none,                             % A frame around the code
tabsize=2,                              % Default tab size
captionpos=t,                           % Caption-position = bottom
breaklines=true,                        % Automatic line breaking?
breakatwhitespace=false,                % Automatic breaks only at whitespace?
showspaces=false,                       % Dont make spaces visible
showtabs=false,                         % Dont make tabls visible
columns=flexible,                       % Column format
morekeywords={__global__, __device__},  % CUDA specific keywords
}

%%%%%%%%%%%%%%%%%%%%%%%%%%%%%%%%%%%%
\begin{document}
\begin{center}
{\Large \textsc{POLS 8500: Causal Inference}}
\end{center}
\begin{center}
{\large Spring 2023}
\end{center}

\begin{center}
\rule{6.5in}{0.4pt}
\begin{minipage}[t]{.96\textwidth}
\begin{tabular}{llcccll}
\textbf{Professor:} & Joe Ornstein & & &  & \textbf{Time:} & Th 3:55pm -- 6:40pm \\
\textbf{Email:} &  \href{mailto:jornstein@uga.edu}{jornstein@uga.edu} & & & & \textbf{Place:} & 101B Baldwin Hall\\
\textbf{Website:} & \href{https://github.com/joeornstein/causal-inference}{github.com/joeornstein/causal-inference} & & & & &
\end{tabular}
\end{minipage}
\rule{6.5in}{0.4pt}
\end{center}
\vspace{.15cm}
\setlength{\unitlength}{1in}
\renewcommand{\arraystretch}{2}

Correlations alone are rarely satisfying. As social scientists, we don't just want to identify patterns in data -- we want to convincingly explain \textit{why} those patterns exist. In this class, we introduce the workhorse methods for inferring causal relationships from observational data, with a focus on building the skills you need to develop convincing research designs, identify useful variation in your data, and write error-free, reproducible code.

\begin{figure}[h]
	\centering
	\href{https://xkcd.com/552/}{\includegraphics[width = 0.7\textwidth]{img/correlation.png}}
\end{figure}

\section*{Requirements}

Your course grade will be based on three components.

\begin{itemize}
	\item \textbf{Readings and Annotations (25\%):} Each week, read the assigned chapters/articles and annotate using \href{https://hypothes.is/groups/Qm2VnAae/causal-inference}{Hypothesis}.
	\item \textbf{Replications (2 $\times$ 25\%):} Choose two of \textit{Optional Bonus Fun} papers on the syllabus below (each student must choose a different set of papers). Using their posted replication materials, recreate the paper's key results in a Quarto report. You will present your findings in a Mock Conference Presentation during class.
	\item \textbf{Final Paper (25\%):} Write a final paper using one or more of the designs we introduced in class. This could be either an original research project or a re-analysis and extension of one of the papers you replicated.
\end{itemize}


\section*{Course Outline}

Here is the rough plan of action for our semester together. Readings abbreviated ``NHK'' denote our primary textbook, Nick Huntington-Klein's \textit{The Effect}, available in hard copy or \href{https://theeffectbook.net/}{free online}.

\subsection*{Week 1: Garbage Cans, Causal Salads, and the Seven Deadly Sins}

\textit{Required Reading:}

\begin{itemize}
	\item Angrist, Joshua D, and Jörn-Steffen Pischke. 2010. ``\href{https://pubs.aeaweb.org/doi/pdfplus/10.1257/jep.24.2.3}{The Credibility Revolution in Empirical Economics: How Better Research Design Is Taking the Con out of Econometrics.}'' \textit{Journal of Economic Perspectives} 24(2): 3–30.
	\item McElreath, Richard. 2021. \href{https://elevanth.org/blog/2021/06/15/regression-fire-and-dangerous-things-1-3/}{\textit{Regression, Fire, and Dangerous Things.}}
	\item Schrodt, Philip A. 2014. ``\href{https://journals.sagepub.com/doi/pdf/10.1177/0022343313499597}{Seven Deadly Sins of Contemporary Quantitative Political Analysis.}'' \textit{Journal of Peace Research} 51(2): 287–300.
\end{itemize}

\subsection*{Week 2: The Fundamental Problem}

\textit{Required Reading:}

\begin{itemize}
	\item NHK Chapters 1-4
\end{itemize}

\subsection*{Weeks 3: D'ya Like DAGs?}

\textit{Required Reading:}

\begin{itemize}
	\item NHK Chapters 5-8
\end{itemize}

\noindent \textit{Deeper Dives:}

\begin{itemize}
	\item Pearl, Judea, and Dana Mackenzie. \textit{The Book of Why: The New Science of Cause and Effect. Penguin Science}. London: Penguin Books, 2019.
\end{itemize}

\subsection*{Week 4: Experiments (Both Natural and Unnatural)}

\textit{Required Reading:}

\begin{itemize}
	\item NHK Chapters 9-11
\end{itemize}

\noindent \textit{Optional Bonus Fun:}

\begin{itemize}
	\item Montgomery, Jacob M., Brendan Nyhan, and Michelle Torres. ``How Conditioning on Posttreatment Variables Can Ruin Your Experiment and What to Do about It.'' \textit{American Journal of Political Science} 62, no. 3 (2018): 760–75. https://doi.org/10.1111/ajps.12357.
\end{itemize}

\subsection*{Week 4: Regression (Omniscience + Linearity)}

\textit{Required Reading:}

\begin{itemize}
	\item NHK Chapter 13
\end{itemize}


\noindent \textit{Optional Bonus Fun:}

\begin{itemize}
	\item Cinelli, Carlos, and Chad Hazlett. ``Making Sense of Sensitivity: Extending Omitted Variable Bias.'' \textit{Journal of the Royal Statistical Society: Series B (Statistical Methodology)} 82, no. 1 (2020): 39–67. https://doi.org/10.1111/rssb.12348.
\end{itemize}

\subsection*{Week 5: Matching (Omniscience without Linearity)}

\textit{Required Reading:}

\begin{itemize}
	\item NHK Chapter 14
\end{itemize}

\noindent \textit{Optional Bonus Fun:}

\begin{itemize}
	\item Hainmueller, Jens. 2012. ``Entropy Balancing for Causal Effects: A Multivariate Reweighting Method to Produce Balanced Samples in Observational Studies.'' \textit{Political Analysis} 20(1): 25–46.
\end{itemize}

\subsection*{Week 6-7: Fixed Effects}

\textit{Required Readings:}

\begin{itemize}
	\item NHK Chapter 16
\end{itemize}


\noindent \textit{Deeper Dives:}

\begin{itemize}
	\item Hazlett, Chad, and Leonard Wainstein. ``Understanding, Choosing, and Unifying Multilevel and Fixed Effect Approaches.'' \textit{Political Analysis} 30, no. 1 (January 2022): 46–65.
	\item Imai, Kosuke, and In Song Kim. ``On the Use of Two-Way Fixed Effects Regression Models for Causal Inference with Panel Data.'' \textit{Political Analysis} 29, no. 3 (July 2021): 405–15.
\end{itemize}

\subsection*{Week X-Y: Difference-in-difference}

\textit{Deeper Dives:}

\begin{itemize}
	\item Roth, Jonathan, Pedro H. C. Sant’Anna, Alyssa Bilinski, and John Poe. ``What’s Trending in Difference-in-Differences? A Synthesis of the Recent Econometrics Literature.'' \textit{arXiv}, January 13, 2022. http://arxiv.org/abs/2201.01194.
	\item https://www.youtube.com/user/ariesxyq/videos
\end{itemize}



\subsection*{Week X-Y: Regression Discontinuity}

\textit{Required Readings:}

\begin{itemize}
	
	\item Cattaneo, Matias D, Nicolas Idrobo, and Rocío Titiunik. 2018. ``A Practical Introduction to Regression Discontinuity Designs: Volume I.''
	\item Marshall, John. 2022. ``Can Close Election Regression Discontinuity Designs Identify Effects of Winning Politician Characteristics?'' \textit{American Journal of Political Science}.
	\item Ornstein, Joseph T. \& Duck-Mayr, JBrandon. 2022. ``Gaussian Process Regression Discontinuity''. \textit{Working Paper}.
\end{itemize}

\noindent \textit{Optional Bonus Fun:}

\begin{itemize}
	\item Klašnja, Marko, and Rocío Titiunik. 2017. ``The Incumbency Curse: Weak Parties, Term Limits, and Unfulfilled Accountability.'' \textit{The American Political Science Review} 111(1): 129–48.
	\item Keele, Luke J., and Rocío Titiunik. 2015. ``Geographic Boundaries as Regression Discontinuities.'' \textit{Political Analysis} 23(1): 127–55.
	
\end{itemize}


\section*{Academic Honesty}

Remember that when you joined the University of Georgia community, you agreed to abide by a code of conduct outlined in the academic honesty policy called \href{https://honesty.uga.edu/Academic-Honesty-Policy/Introduction/}{\textit{A Culture of Honesty}}. It has some pretty specific things to say on the subject of cheating. Quite specific. I will report any and all dishonest conduct to the Office of the Vice President for Instruction.

\section*{Mental Health and Wellness Resources}

\begin{itemize}
	\item If you or someone you know needs assistance, you are encouraged to contact Student Care and Outreach in the Division of Student Affairs at 706-542-7774 or visit \href{https://sco.uga.edu}{https://sco.uga.edu}. They will help you navigate any difficult circumstances you may be facing by connecting you with the appropriate resources or services. 
	\item UGA has several resources for a student seeking \href{https://www.uhs.uga.edu/bewelluga/bewelluga}{mental health services} or \href{https://www.uhs.uga.edu/info/emergencies}{crisis support}. 
	\item If you need help managing stress anxiety, relationships, etc., please visit \href{https://www.uhs.uga.edu/bewelluga/bewelluga}{BeWellUGA} for a list of FREE workshops, classes, mentoring, and health coaching led by licensed clinicians and health educators in the University Health Center.
	\item Additional resources can be accessed through the UGA App.
\end{itemize}


%%%%%% THE END 
\end{document} 